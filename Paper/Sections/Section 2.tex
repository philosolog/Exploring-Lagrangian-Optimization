\lesson{2}{}{The Extreme Value Theorem in $\mathbb{R}^3$}
% *: 
\setcounter{chapter}{2}
\chapter{Hungrier Joe}
Since Joe is a math aficionado, he had already mentally precomputed that he needed $4$ units of sweetness in order to achieve his maximum satisfaction of $8$ utils.
Because of this, Joe was fixated on a far more troubling matter...

Like other ice cream parlors, Carl's Parlor serves high-quality vegetable-based chicken strips as an ice cream topping.
Unfortunately, that is the ONLY topping at Carl's.

Joe ponders the most optimal combination of cotton candy ice cream and chicken strips that will provide him with the maximum satisfaction.
Joe's satisfaction $S$ can now be represented in terms of sweetness $(s)$ and umami $(u)$ as:\par
\LARGE
\begin{equation}
	S(s, u) = 8e^{-\frac{(s-4)^2+(u-4)^2}{64}}
\end{equation}
\normalsize
\\
\begin{eg}
	Joe desires for at least $0$ units of either taste and a total sum of tastes that does not exceed $16$ units.

	What is the maximum satisfaction that Joe can achieve?
\end{eg}

\setcounter{chapter}{3}
\chapter{Nerd Face Emoji}
Being the second-to-highest-gold-star-sticker student in Mr. Barraza's multivariable calculus class, Joe figured that he would have to use the \textbf{Extreme Value Theorem in $\mathbb{R}^3$} to solve this problem.
\begin{theorem}[The Extreme Value Theorem in $\mathbb{R}^3$, Paul's Online Notes]
	If \(f\left( {x,y} \right)\) is continuous in some closed, bounded set \(D\) in \({\mathbb{R}^2}\) then there are points in \(D\), \(\left( {{x_1},{y_1}} \right)\) and \(\left( {{x_2},{y_2}} \right)\) so that \(f\left( {{x_1},{y_1}} \right)\) is the absolute maximum and \(f\left( {{x_2},{y_2}} \right)\) is the absolute minimum of the function in \(D\).
\end{theorem}
The EVT in $\mathbb{R}^3$ is similar to the EVT in $\mathbb{R}^2$ except that, in order for the theorem to apply, the inputs $(x, y)$ to a $\mathbb{R}^2$ function $f$ must exist in a closed and bounded region.
If the latter case suffices, the EVT states that there exists absolute extrema for a function $f$ in such region.

Firstly, we have to find the $3$D critical points for the function $S$.
In order to do this, we must find where the partial derivatives of the function equal zero.
\begin{align*}
	\frac{\partial}{\partial s} = -\frac{1}{4}(u-4)e^{-\frac{(s-4)^2+(u-4)^2}{64}}=0\\
	\frac{\partial}{\partial u} = -\frac{1}{4}(s-4)e^{-\frac{(s-4)^2+(u-4)^2}{64}}=0
\end{align*}
From a similar expression (see Chapter 2, Utilmaxxing), we can simplify this to see that:
\begin{align*}
	s = 4\text{; }u = 4
\end{align*}
Doing some algebra and finding the corresponding values for each variable's solution in $(s, u)$ will yield one critical point solution, $(4, 4)$.

Then, we have to test for points of absolute extrema on the bounds of the region.
Here, we'll use what we know about the EVT in $\mathbb{R}^2$ to test for absolute extrema.
Though, we must first describe the bounding functions!

The problem implies the following relations:
\begin{enumerate}
	\item $s\geq 0$
	\item $u\geq 0$
	\item $s+y\leq 16$
\end{enumerate}

To find a single variable function for each bound, given the satisfaction function, we must make some substitutions at the extreme/boundary cases...
\begin{center}
	$(s=0\text{; }u=0\text{; }s+u=16)$
\end{center}

For relation $1$, we get:
\begin{align*}
	S(0, s) = 8e^{-\frac{(-4)^2+(u-4)^2}{64}}
\end{align*}
% *: 
\setcounter{chapter}{4}
\chapter{Metonymization, Part 1}
Now, let's try a general example that requires the application of the Extreme Value Theorem in $\mathbb{R}^3$.
\begin{eg}
	Find the absolute minimum and absolute maximum of
	\begin{center}
		$f(x, y) = x^2 - y^2 + xy - 5x$
	\end{center}
	on the region bounded by $y = 5 - x^2$ and the $x$-axis.
\end{eg}
% TODO: More applications...