\lesson{2}{}{The Extreme Value Theorem in $\mathbb{R}^3$}
% *: 
\setcounter{chapter}{2}
\chapter{Hungrier Joe}
Since Joe is a math aficionado, he had already mentally precomputed that he needed $4$ units of sweetness in order to achieve his maximum satisfaction of $8$ utils.
Because of this, Joe was fixated on a far more troubling matter...

Like other ice cream parlors, Carl's Parlor serves high-quality vegetable-based chicken strips as an ice cream topping.
Unfortunately, that is the ONLY topping at Carl's.

Joe ponders the most optimal combination of cotton candy ice cream and chicken strips that will provide him with the maximum satisfaction.
Joe's satisfaction $S$ can now be represented in terms of sweetness $(s)$ and umami $(u)$ as:\par
\LARGE
\begin{equation}
	S(s, u) = 8e^{-\frac{(s-4)^2+(u-4)^2}{64}}
\end{equation}
\normalsize
\\
\begin{eg}
	Joe desires for at least $0$ units of either taste and a total sum of tastes that does not exceed $16$ units.

	What is the maximum satisfaction that Joe can achieve?
\end{eg}

\setcounter{chapter}{3}
\chapter{Nerd Face Emoji} % From Mr. Barraza's class, lol... Also, add emojis here...
\begin{theorem}[$^\dagger$The Extreme Value Theorem in $\mathbb{R}^3$]
	
\end{theorem}
% *: 
\setcounter{chapter}{4}
\chapter{Metonymization, Part 1}
Now, let's try a general example that requires the application of the Extreme Value Theorem in $\mathbb{R}^3$.
\begin{eg}
	Find the absolute minimum and absolute maximum of
	\begin{center}
		$f(x, y) = x^2 - y^2 + xy - 5x$
	\end{center}
	on the region bounded by $y = 5 - x^2$ and the $x$-axis.
\end{eg}
% TODO: More applications...