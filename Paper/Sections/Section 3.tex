\lesson{3}{}{The Method of Lagrange Multipliers}
% *: 
\setcounter{chapter}{5}
\chapter{Another Order}
Delightful! Joe greatly enjoyed the addition of meat- the piquant umami was a new experience for his buds.

Despite already heightening his satisfaction twice, Joe was yet again deciding on another combination of a meat-topped ice cream.
This time, his only constraint is that he wants the umami flavor to be inversely proportional to half the sweetness felt.
\begin{eg}
	Given that his satisfaction can again be represented by eq. (3.1), Joe desires for a nonnegative amount of each flavor and wants to try a combination where the flavor of umami he attains is inversely proportional to half the sweetness.

	Can you find a combination of $(s, u)$ where Joe can attain his maximum satisfaction?
\end{eg}
% *: 
\setcounter{chapter}{6}
\chapter{A Joe Analysis}
To solve this problem, we must utilize Lagrange multipliers in order to find the extrema of an arbitrary function $f$ constrained by another function that we'll call $g$.

The method of Lagrange multipliers for solving constrained optimization problems utilizes the following formula:
\large
\begin{equation}
	\vec{\nabla}f(x, y) = \lambda \vec{\nabla}g(x, y)
\end{equation}
\normalsize
In the above equation, $\vec{\nabla}$ represents the gradient operator on a function, which produces a vector where each component represents the partial derivative of the function with respect to the corresponding variable.
This formula displays that the gradient vector of $f$ is equal to the gradient vector of $g$ times an unknown multiple $\lambda$.
$\lambda$ is called a Lagrange multiplier and helps constrain our possible solutions such that $f$ is a multiple of $g$.
If we were to separate this vector equation into partially differentiated functions with respect to either $x$ and $y$ components, we'd get the following equations:
\begin{center}
	$f_x=\lambda g_x$\\
	$f_y=\lambda g_y$
\end{center}
Upon solving for $x$ and $y$ in this equation, you'd get the values of $x$ and $y$ that are candidates for absolute extrema outputs of the function $f$.
Now, let's help Joe.

\pagebreak
We are tasked with optimizing a function $S$ constrained to $u=\frac{2}{s}$. We can describe our constraint function $g$ to be:
\begin{equation}
	g(s, u) = su-2
\end{equation}
Taking some partial derivatives and rewriting the vectors to be equations for each component:
\begin{align*}
	-\frac{1}{4}(u-4)e^{-\frac{(s-4)^2+(u-4)^2}{64}} &= \lambda u\\
	-\frac{1}{4}(s-4)e^{-\frac{(s-4)^2+(u-4)^2}{64}} &= \lambda s
\end{align*}
Okay, we have $3$ unknowns.. but only $2$ equations.
To address this, we will utilize our constraint equation.
\begin{align*}
	-\frac{1}{4}(u-4)e^{-\frac{(s-4)^2+(u-4)^2}{64}} &= \lambda u\\
	-\frac{1}{4}(s-4)e^{-\frac{(s-4)^2+(u-4)^2}{64}} &= \lambda s\\
	su = 2
\end{align*}
Now, we can use basic algebra to find solutions for $(s, u, \lambda)$. Because we are only concerned with which $(s, u)$ that produce the maximum $S$ for Joe, we will ignore using $\lambda$ once found.

Because the algebra will be gruesome to type out for this calculus-based research paper, here is the solution for $(s, u)$:
\begin{align*}
	\left(\frac{8}{(2\sqrt{2}-1)e^{\frac{1}{2\sqrt{2}}-\frac{9}{16}}}, \frac{1}{4}(2\sqrt{2}-1)e^{\frac{1}{2\sqrt{2}}-\frac{9}{16}}\right) \approx (5.39208, 0.37091)
\end{align*}
If you were to do this by hand, you'd find a solution with a $u$ value that is less than $0$.
Because Joe wants a nonnegative amount of each flavor, we can omit the corresponding solution.
% *: 
\setcounter{chapter}{7}
\chapter{Metonymization, Part 2} % Is there a better to word "an off-example"?
The method of Lagrange multipliers for constrained optimization is useful for finding where the extrema of $3$D functions occur, especially if the constraint functions are unbounded.
This is because the method utilizing EVT in $\mathbb{R}^3$, as mentioned in a previous chapter, cannot suffice for unbounded constraints.
Do note that the method of Lagrange multipliers will find the maximum of a function.
Finding the minimum would be finding the maximum of the negated function.

Now, for what this project was all about, let's apply Joe's strategy to a general example...
\begin{eg}
	Use the method of Lagrange multipliers to find the maximum of $2x-y$ given the following constraint:
	\begin{center}
		$3x^2-4xy+2y^2=6$
	\end{center}
	Also, find the point $(x, y)$ where said maximum is achieved.
\end{eg}
Take the partial derivatives and set the equations up:
\begin{align*}
	2 = \lambda (6x-4y)\\
	-1 = \lambda (-4x+4y)\\
	3x^2-4xy+2y^2=6
\end{align*}
We'll leave the algebra as an exercise to the meticulous reader.
Here is the candidate for absolute extrema:
\begin{align*}
	(x, y, \lambda): \left(2, 1, \frac{1}{4}\right)
\end{align*}

Therefore, the maximum of $f$ is $\boxed{3\text{ at }(2, 1)}$.